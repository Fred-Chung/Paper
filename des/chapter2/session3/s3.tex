\section{Bases in Banach spaces}\label{bases}
%%%%%%%%%%%%


The definition and usefulness of a basis in a finite dimensional vector space is
well-known. It is natural then to want an analogous concept for
Banach spaces. The most useful approach is found in the notion of a Schauder
basis. A comprehensive introduction to such bases is \cite[pp. 1--52]{Lind},
upon which most of the material in this section is based.

\begin{definition}
A sequence $\{x_n\}_{n=1}^{\infty}$ in a Banach space $\X$ is called a {\em
Schauder basis of $\X$} if, for every $x\in\X$, there is a unique sequence of
scalars $\{a_n\}_{n=1}^{\infty}$ such that $x=\sum_{n=1}^{\infty}a_nx_n$. A
sequence $\{x_n\}_{n=1}^{\infty}$ which is a Schauder basis for its closed
linear span is called a {\em basic sequence}.
\end{definition}

A Banach space $\X$ is called {\em separable} if it has a countable dense
subset.
Thus if $\X$ has a Schauder basis then $\X$ is separable. In general the
converse
is not true. Since the class of Schauder bases is the only type of bases
considered in this
thesis, it will henceforth be implicit that any discussion about bases refers
only to Schauder bases.


\begin{example}
The ordered set of unit vectors $e_n=(0,0,0,\ldots,1,0,\ldots)$, where the $1$
occurs in the $n$th coordinate, forms a basis in each of the spaces
$c$, $c_0$ and $\lp$. Another example of a basis in $c$ is given by
$x_1=(1,1,1,\ldots)$
and $x_n=e_{n-1}$ for $n>1$. If $x=(b_1,b_2,\ldots)\in c$, then the expansion of
$x$ with respect to this basis is
\[x=(\lim_{k\rightarrow\infty}a_k)x_1+(a_1-\lim_{k\rightarrow\infty}a_k)x_2+
(a_2-\lim_{k\rightarrow\infty}a_k)x_3+\cdots.\]
\end{example}


If a Banach space $\X$ has a basis, we may consider $\X$ as a sequence
space via an identification of each element $x=\sum_{n=1}^{\infty}a_nx_n$ in
$\X$ with the unique sequence of coefficients $(a_1,a_2,a_3,\ldots)$. To do this
note that it is essential to describe the basis as an ordered sequence rather
than merely as a set. The following example highlights a deeper reason for
why we must specify the ordering of the vectors in a basis.

\begin{example}\label{summing basis}
Consider the sequence of vectors $\{x_n\}_{n=1}^{\infty}$ in $c$ given by
\[x_n=(\underbrace{0,0,\ldots,0}_{n-1},1,1,\cdots),\]
for $n\in\Z^+$. It is not hard to see that this is a basis for $c$.
It is called the {\em summing basis} in $c$.
Consider the sequence $b$ defined by
\[b_n=\sum_{k=1}^n(-1)^{k+1}\frac{1}{k}\]
for $n\in\Z^+$. By Leibniz' test for alternating series,
$b_n$ has a limit as $n\rightarrow\infty$, so $b\in c$. The expansion of $b$
with
respect to the summing basis is $b=\sum_{n=1}^{\infty}a_nx_n$ where
$a_n=(-1)^{n+1}\frac{1}{n}$. Now consider the following rearrangement of the
series. Define $\pi:\Z^+\rightarrow\Z^+$ by
\[\pi(n)=\left\{\begin{array}{ll}
				2k & \mbox{if $n=3k$}\\
				4k+1 & \mbox{if $n=3k+1$}\\
				4k+3 & \mbox{if $n=3k+2$.}
			\end{array}\right.
\]
Then $\pi$ is clearly a permutation of $\Z^+$. However, the expansion
\[b'\equiv\sum_{n=1}^{\infty}a_{\pi(n)}x_{\pi(n)}\]
does not lie in $c$ since
\[b_{3n}'=\sum_{k=1}^n\left(\frac{1}{2k-1}+\frac{1}{2k+1}-\frac{1}{2k}\right)
\geq 2\sum_{k=1}^n\frac{1}{k}\]
diverges as $n\rightarrow\infty$.
\end{example}

Not every basis has the defect that the convergence of an expansion with respect
to the basis is dependent on the order of summation of the expansion. For
example, expansions in finite dimensional spaces can be summed in any order
without altering convergence. The following proposition gives various methods
for
checking whether a sum of vectors can be freely rearranged while still
respecting convergence.

\begin{proposition}\label{unconditional conv}
\cite[Proposition 1.c.1]{Lind}
Let $\{x_n\}_{n=1}^{\infty}$ be a sequence of vectors in a Banach space $\X$.
Then the following are equivalent.

(i) The series $\sum_{n=1}^{\infty}x_{\pi(n)}$ converges for every permutation
$\pi$ of the integers.

(ii) The series $\sum_{k=1}^{\infty}x_{n_k}$ converges for every choice of
$n_1<n_2<n_3\ldots$.

(iii) The series $\sum_{n=1}^{\infty}\varepsilon_nx_n$ converges
for every choice of signs $\varepsilon_n=\pm 1$.

(iv) For every $\epsilon>0$ there exists an integer $n_0$ such that
$\ssnorm{\sum_{j\in J}x_j}_{\X}<\epsilon$ for every finite set of integers $J$
which satisfies $\min\{j\in J\}>n_0$.
\end{proposition}

With the aid of the proposition, there are now easier ways to verify that the
series
$\sum_{n=1}^{\infty}a_nx_n$ given in Example \reff{bases}{summing basis} is
dependent on the order of summation. Simply observe that
$\sum_{n=1}^{\infty}a_{2n-1}x_{2n-1}$ does not converge in $c$ and hence
statement (ii) of the proposition fails to hold.

\begin{definition}
Let $\{x_n\}_{n=1}^{\infty}$ be a sequence of vectors in a Banach space $\X$.
A series $\sum_{n=1}^{\infty}x_n$ which satisfies any of the conditions (i),
(ii), (iii) or (iv) in Proposition \reff{bases}{unconditional conv} is said to
be
{\em unconditionally convergent}. A basis $\{x_n\}_{n=1}^{\infty}$ of a Banach
space $\X$ is {\em unconditional} if for every $x\in\X$, its expansion
$x=\sum_{n=1}^{\infty}a_nx_n$ converges unconditionally. Otherwise the basis is
said to be {\em conditional}.
\end{definition}

Thus the standard ordered bases $\{e_n\}_{n=1}^{\infty}$ for $c$, $c_0$ and
$\lp$,
$1\leq p<\infty$ are unconditional bases. This means that any reordering of
$\{e_n\}_{n=1}^{\infty}$ also yields unconditional bases in these spaces.
The summing basis for $c$ is clearly a conditional basis.

Given a basis $\{x_n\}_{n=1}^{\infty}$, one might ask what
complex sequences $\{\phi_n\}_{n=1}^{\infty}$ give rise to a bounded multiplier
transform? That is, is there a constant $C>0$ such that for all $x\in\X$,
\[\snorm{\sum_{n=1}^{\infty}\phi_na_{n,x}x_n}_{\X}\leq C\norm{x}_{\X},\]
where $\sum_{n=1}^{\infty}a_{n,x}x_n$ is the expansion of each $x\in\X$ with
respect to the basis? If that basis is unconditional, then the space of such
sequences is just $\ell^{\infty}$ (see \cite[Proposition 1.c.7]{Lind}).


We have seen that conditional bases do exist, and some of the resulting
difficulties that arise when working with a conditional basis. Is there any way
in which we can avoid such problems in separable Banach spaces? In other words,
can we always find an unconditional
basis for a separable Banach space? The answer is no. The space $L^1[0,1]$ has
no unconditional basis. In fact, $L^1[0,1]$ cannot even be embedded in a space
with an unconditional basis \cite[Propostion 1.d.1]{Lind}.

There are many open problems regarding unconditional bases. For example,
suppose $\X$ is a Banach space with an unconditional basis, and let $\Y$ be
a complemented subspace of $\X$. Does $\Y$ have an unconditional basis? One
outstanding problem has recently been solved (see \cite{Gowers}).
Does every infinite dimensional Banach space $\X$ contain an unconditional
basic sequence? The answer is no.
We shall not pursue such broad problems in this thesis. Instead, we shall use
tools from harmonic analysis and spectral theory to examine whether or not
particular sequences in certain Banach spaces are conditional, and whether or
not
they form bases for those spaces.
