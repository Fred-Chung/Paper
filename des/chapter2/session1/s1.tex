\section{Banach, Operator and Dual Spaces}\label{Banach Spaces}
%%%%%%%%%%%



We recall some basic definitions and concepts from Banach space theory and
functional analysis. A good introductory reference for this material is
\cite{TEST}.

A {\em Banach space} $\X$ is a vector space $V$ over a field $\F$ that is
equipped with a norm
$\norm{\,\cdot\,}_{\X}$ that makes $V$ complete with respect to the metric $d$
given by $d(x,x')=\norm{x-x'}_{\X}$. In this thesis we shall always assume that
the
underlying field $\F$ of scalars is $\C$. Suppose $\Y$ is also a Banach space
with norm $\norm{\,\cdot\,}_{\Y}$. A linear operator
$T:\X\rightarrow\Y$ is said to be {\em bounded} if
$\sup\{\norm{Tx}_{\Y}:x\in\X,\norm{x}_{\X}=1\}<\infty$. The collection of
$\Y$-valued bounded linear operators on $\X$ is denoted $\B(\X,\Y)$, and is
itself a Banach space when equipped with the {\em operator norm}
$\norm{T}=\sup\{\norm{Tx}_{\Y}:x\in\X,\norm{x}_{\X}\leq 1\}$. A $\Y$-valued
linear
operator on $\X$ is bounded if and only if it is continuous with respect to the
topologies on $\X$ and $\Y$ induced by their respective norms. Furthermore, if
$T\in\B(\X,\Y)$ then $\norm{Tx}_{\Y}\leq\norm{T}\norm{x}_{\X}$ for all $x\in\X$.

Henceforth, the norm of a Banach space $\X$ will be denoted
$\norm{\,\cdot\,}_{\X}$ (except when $\X$ is an $L^p$ space --- see the
example below), while the operator norm $\norm{\,\cdot\,}$ will
contain no subscript because it is clear from the operator's definition what
spaces its norm is induced from. The space $\B(\X,\X)$ of bounded linear
operators sending elements of $\X$ into $\X$ will be denoted $\B(\X)$. An
important class of operators included in $\B(\X)$ are the {\em projections} on
$\X$, those bounded linear maps $P$ satisfying $P^2=P$.

There are many examples of Banach spaces. We give one example that will feature
regularly throughout the thesis.

\begin{example}\label{L^p example}
Let $(\Omega,\A,\mu)$ be a measure space and $1\leq p<\infty$. Let
$\X=L^p(\Omega,\A,\mu)$ be the space of all equivalence classes of
$\C$-valued $\A$-measurable functions $f$ on $\Omega$ with the property that
$\int_{\Omega}|f(x)|^p\,d\mu(x)<\infty$. We shall usually blur the distinction
between the equivalence classes of $L^p(\Omega,\A,\mu)$ and the representatives
of these classes. The norm on $\X$ is given by
$\norm{f}_{\X}=(\int_{\Omega}|f(x)|^p\,d\mu(x))^{1/p}$ and
makes $\X$ a Banach space. When the underlying $\sigma$-algebra and measure is
standard (for example, when $\Omega=\R$, $\A$ is the Borel $\sigma$-algebra
and $\mu$ is Lebesgue measure on $\R$), we shall denote
$L^p(\Omega,\A,\mu)$ by $L^p(\Omega)$.
Because it
is notationally less cumbersome to write $\norm{\,\cdot\,}_p$ instead of
$\norm{\,\cdot\,}_{L^p(\Omega,\A,\mu)}$, we shall do so whenever the context
permits no ambiguity. A particularly useful fact is that if $(\Omega,\A,\mu)$ is
a finite measure space and $1\leq p<r<q\leq\infty$, then
$L^p(\Omega)\cap L^q(\Omega)\subset L^r(\Omega)$ and
$L^q(\Omega)\subset L^p(\Omega)$. A quick source of information on these spaces
is \cite[\S 6.4]{Pedersen}.
\end{example}

\begin{example}
We give a few of the sequence spaces.
\begin{enumerate}
\item For $1\leq p<\infty$, define $\lp$ to be
the space of all functions $f:\N\rightarrow\C$ such that
$\sum_{n=1}^{\infty}|f(n)|^p<\infty$, and equip it with the norm
$\norm{f}_{\lp}=(\,\sum_{n=1}^{\infty}|f(n)|^p\,)^{1/p}$. Thus $\lp$ may be
identified with $L^p(\N)$, where the underlying measure is counting measure on
$\N$.
\item The space $\ell^{\infty}$ is the set of all bounded functions
$f:\N\rightarrow\C$ with norm $\norm{f}_{\ell^{\infty}}=\sup\{|f(n)|:n\in\N\}$.
\item The subset $c$ of $\ell^{\infty}$ of all convergent sequences
has norm given by $\norm{f}_c=\norm{f}_{\ell^{\infty}}$ for $f\in c$.
\item The subset $c_0$ of $\ell^{\infty}$ of all bounded functions converging to
zero has norm given by $\norm{f}_{c_0}=\norm{f}_{\ell^{\infty}}$ for $f\in c_0$.
\end{enumerate}These are all Banach spaces.
\end{example}

To every Banach space is associated another Banach space known as its dual.
Suppose $\X$ is a Banach space. A {\em linear functional} on $\X$ is a linear
map from $\X$ into $\C$. The set of all bounded linear functionals on $\X$,
denoted
by $\X^*$, is made a vector space via pointwise operations. We equip $\X^*$ with
a norm given by $\norm{x^*}_{\X^*}=\sup\{|x^*(x)|:x\in\X,\norm{x}\leq 1\}$.
Thus $\X^*$ coincides with $\B(\X,\C)$ and is itself a Banach
space. We call $\X^*$ the {\em dual space} of $\X$. Given $x^*\in\X^*$ and
$x\in\X$, it is standard to write
$\ip<x,x^*>$ for $x^*(x)$.

It can be shown that for $1<p<\infty$, the dual space of $\X=L^p(\Omega,\A,\mu)$
is isometrically isomorphic to $\X=L^q(X,\Omega,\mu)$, where $q$ satisfies
$1/p+1/q=1$. Thus $(\lp)^*$ is isomorphic to $\ell^q$ as a Banach space. It is
also known that $c_0^*=\ell^1$ and $(\ell^1)^*=\ell^{\infty}$. See
\cite[Chapter III, \S 5 and \S 11]{Con} for more details.

Given a Banach space $\X$, we may take its dual $\X^*$. This is also a Banach
space and hence has a dual $(\X^*)^*$, called {\em the second dual of $\X$}
and written $\X^{**}$. Continuing in this way we may construct a sequence
$\X,\X^*,\X^{**},\X^{***},\ldots$ of Banach spaces. For which
spaces $\X$ does this list contain any repetitions? This question leads us to
consider an important class of Banach spaces, known as the reflexive Banach
spaces.

Let $\X$ be Banach space. To each $x\in\X$ we may associate a unique element
$\hatt{x}\in\X^{**}$ by the rule $\hatt{x}(x^*)=\ip<x,x^*>$ for all $x^*\in\X$.
The map $x\mapsto\hatt{x}$ from $\X$ into $\X^{**}$ is called the {\em natural
map} of $\X$ into its second dual.

A Banach space $\X$ is said to be {\em reflexive} if
$\X^{**}=\{\hatt{x}:x\in\X\}$. If $\X$ is reflexive then its second dual
$\X^{**}$ is isometrically isomorphic to $\X$. However, the converse does not
hold (see \cite[III.11]{Con}). It is a consequence of the Riesz Representation
Theorem \cite[I.3.4]{Con} that every Hilbert space is reflexive.
The spaces $\X=L^p(X,\Omega,\mu)$ and
$\lp$ are also reflexive for $1<p<\infty$, but by the above remarks $c_0$ is
not. Since
$c_0^{**}=\ell^{\infty}$, it is clear that $c_0\subset c_0^{**}$. In fact,
$c_0$ is isometrically embedded into its second dual $\ell^{\infty}$. This fact
generalises to
all Banach spaces. The space $C[0,1]$ of continuous functions on $[0,1]$ with
supremum norm is another example of a non-reflexive Banach space.

A {\em Banach algebra} $\mathfrak{A}$ is an algebra over $\F$ that has a norm
$\norm{\,\cdot\,}_{\mathfrak{A}}$ relative to which $\mathfrak{A}$ is a Banach
space and
\[\norm{ab}_{\mathfrak{A}}\leq\norm{a}_{\mathfrak{A}}\norm{b}_{\mathfrak{A}}\]
for all $a,b\in\mathfrak{A}$. If $\mathfrak{A}$ has an identity $e$, then it is
assumed that $\norm{e}_{\mathfrak{A}}=1$. In this thesis we shall always take
$\F$ to be $\C$. An example of a Banach algebra is the space $C[0,1]$ equipped
with the supremum norm, with multiplication of elements in $C[0,1]$ defined
pointwise.
