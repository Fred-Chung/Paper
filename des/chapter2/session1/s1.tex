\section{Bayesian Theorem}\label{Banach Spaces}
We can never be completely sure of this world, because it is a constantly changing being, and change is the essence of reality. However, what we can do is, as expressed by this theorem, as we obtain more and more data or evidence, our knowledge of reality has been updated and improved
\begin{eqnarray*}
P(A|B) &=& \frac{P(A,B)}{P(B)} \\
& = &\frac{P(B|A)*P(A)}{P(B)} \\
\end{eqnarray*}
\begin{itemize}
  \item $P(A | B)$ is the conditional probability of A after the occurrence of B is known, and is also excluded from the posterior probability of A due to the value obtained from B, indicating the confidence that event A will occur after event B occurs.

  \item $P(A)$ is the a priori probability or edge probability of A, and represents the confidence that event A occurs.

  \item $P(B|A)$ is the conditional probability of B after the occurrence of A. It is also called the posterior probability of B because of the value obtained from A, and is also considered as a likelihood function.

  \item $P(B)$ is the prior probability or edge probability of B, which is called a standardized constant.

  \item $P(B | A) P(B)$ is called the standard likelihood ratio (there are many names, and no unified standard name is found), which indicates the degree of support provided by event B for the occurrence of event A.
\end{itemize}
\subsection{Prior Probability}

\subsection{Likelihood Probability}

\subsection{Posterior probability}
