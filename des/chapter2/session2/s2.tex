\section{Basic distribution}

\subsection{Beta distribution}
The beta distribution can represent the probability distribution of a probability. When you don't know the specific probability of a thing, it can be called the probability of the occurrence of all probabilities

Definition The Beta Function,For each positive $\alpha$ and $\beta$.define:
$$P(x|\alpha,\beta) = \frac{1}{B(\alpha,\beta)}x^{\alpha-1}(1-x)^{\beta-1} $$
%
% The function B is called beta function.
$$B(\alpha,\beta) = \frac{\Gamma(\alpha)  \Gamma(\beta)}{\Gamma{\alpha+\beta}}$$

For example,carry out $N$ times of Bernoulli test, the probability of success of the test p is subject to an a priori probability density distribution $Beta(\alpha,\beta)$ , the test result appears $K$ times of success of the test, the posterior probability density distribution of the probability p of the test success is $Beta(\alpha + K,\beta+N-K)$.Prove:

Prior distribution:
\[
  f(p|\alpha,\beta) &=& \frac{\Gamma(\alpha)  \Gamma(\beta)}{\Gamma{\alpha+\beta}}x^{\alpha-1}(1-x)^{\beta-1} \
\]
Likelihood Function:
\[
  f(n_1,n_2,\ndot,N|p) = p^{K}(1-p)^{N-K}
\]
Posterior distribution:
\begin{eqnarray*}
  f(p|n_1,n_2,..N,\alpha,\beta) &=& \frac{f(n_1,n_2,\ndot,N|p)f(p|\alpha,\beta)}
  {f(n1,n2,...N,\alpha,\beta)}
\end{eqnarray*}

Given that:
\begin{eqnarray*}
  f(n1,n2,...N,\alpha,\beta) &=& \int_p f(n_1,n_2,\ndot,N|p)f(p|\alpha,\beta) \\
  & = &\frac{1}{B(\alpha,\beta)} \int_p p^{\alpha+K-1}(1-p)^{\beta+N-K-1} \\
  &=&  \frac{B(\alpha+k,\beta+N-K)}{B(\alpha,\beta)}
\end{eqnarray*}
\begin{eqnarray*}
  f(n_1,n_2,\ndot,N|p)f(p|\alpha,\beta) &=& \frac{1}{B(\alpha,\beta)}x^{\alpha-1}(1-x)^{\beta-1} * \frac{f(n_1,n_2,\ndot,N|p)f(p|\alpha,\beta)}
  {f(n1,n2,...N,\alpha,\beta)} \\
  &=& \frac{1}{B(\alpha,\beta)}p^{\alpha+K-1}(1-p)^{\beta+N-K-1}
\end{eqnarray*}

So that:
\begin{eqnarray*}
  f(p|n_1,n_2,..N,\alpha,\beta) &=& \frac{1}{\alpha+K-1}(1-p)^{\beta+N-K-1}p^{\alpha+K-1}(1-p)^{\beta+N-K-1} \\
  &=& Beta(\alpha + K,\beta+N-K)
\end{eqnarray*}


\subsection{Dirichlet distribution}
Dirichlet distribution is a generalization of the beta distribution in high dimensions
\subsection{Poisson and Multinomal distribution}
\subsection{conclusion}
Taking a coin toss as an example, after you get a coin, normally the probability of a uniform coin appearing on both sides should be the same, both 0.5, but you are not sure about the material and weight distribution, you need to judge its Is it really evenly distributed
