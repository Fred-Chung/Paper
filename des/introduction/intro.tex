\afterpage{\blankpage}

\chapter{Introduction}\label{s-intro}

\section{Background and Importance}
With the development of technology and alomost all information is digitized, we are more and more likely to be overwhelmed by a large amount of information, and it is difficult to find an effective way to find, organize and understand a large amount of information. Understanding text has become a popular research topic for scientist{\cite{intro-back}}. However
% Mention of previous work on the subject
% Statemeng of the importance of the subject

% 引言部分的第一段需要给出研究领域的大背景及其重要性所在。这个大背景勾勒出该领域科研成果从古至今的一个走向或者趋势 (what is known),为接下来本论文课题的发展生长提供温床。这部分内容的展开一定要引用该领域前人、大牛的经典文献或者奠基性著作,体现你对于该学科的一个总体把握是全面且客观的。那如何从这个大背景引入到论文的课题呢?当我们从该学术领域的大趋势逐渐缩小范围时,只需把其中
?
% 摆出该研究领域的一个概况之后,就要顺理成章地指出哪些是我们还未涉及的领域(或是没有研究透彻的问题)。当然这样的问题有千千万万,此处不能全部罗列,必要要对应着你的论文课题来谈。直白地说,就是你论文课题研究的什么,此处就针对性地写“这个课题尚未被太多科研者涉足”云云。

\section{topic of research paper}
Announce the research topic/question being addressed in research paper

% 既然研究空白已近在眼前,那么对应的研究课题便要紧随其后。接下来就是你理直气壮陈述该论文主题的时刻,此时需要注意与研究空白的呼应,即在摆出课题是什么的基础之上更近一层,简要分析这个课题是

\section{Highlight the approach and principal finding}

DEscrition of your paper approach and why you chose it

brief summary of your major findings

\section{Organization of this paper}
