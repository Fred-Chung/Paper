\section{Harmonic Analysis on Locally Compact Abelian Groups}\label{lca}
%%%%%%%%%%%%%%%%

In this section we give the necessary background from which we can begin to
answer
the questions from
above. The theory mentioned here can be found in \cite{Katznelson}
and is quite
standard. In what follows, if $G$ is a group, the inverse of an
element $x\in G$ will be denoted by $-x$, and the group operation from
$G\times G$ into $G$ will be denoted by $(x,y)\mapsto x+y$.


\begin{definition} A {\em locally compact Abelian group} (or an LCA group)
$G$ is an Abelian group
which is also a locally compact Hausdorff space such that the group operations
$x\mapsto -x$ from $G$ onto $G$ and $(x,y)\mapsto x+y$ from $G\times G$
onto $G$ are continuous.
\end{definition}

The most-studied examples of LCA groups are the integers $\Z$, the circle group
$\T$ and real line $\R$ with their usual topologies.
It is easy to verify that any Abelian group can be made into an LCA group
when endowed with the discrete topology. In this thesis the circle group $\T$
will feature most often. It can be modelled by the unit circle
$\{\omega\in\C:|\omega|=1\}$
in the complex plane, or as the quotient group $\R/2\pi\Z$. We shall usually
adopt the latter model,
thinking of $\T$ as the interval $[0,2\pi]$ with endpoints identified and
addition as the group operation.

A good reason to study LCA groups is
that we can use them to construct measure spaces that are translation invariant.

\begin{definition} Let $G$ be a locally compact Abelian group. A {\em Haar
measure} on $G$ is a positive regular Borel measure $\mu$ having the
following two properties:

(i) $\mu(E)<\infty$ if $E\subseteq G$ is compact; and

(ii) $\mu(E+x)=\mu(E)$ for all measurable $E\subseteq G$ and all $x\in G$.
\end{definition}

\begin{theorem}\cite[Chapter VII, \S 2]{Katznelson}
Let $G$ be an LCA group. Then a Haar measure on $G$ exists and is unique up to
multiplication of a positive constant.
\end{theorem}

Hence one often speaks of {\em the} Haar measure. For $G=\T$, Haar measure is
usually taken to be normalised
Lebesgue measure, $(2\pi)^{-1}dt$. If $G$ is discrete and infinite, Haar measure
is usually
normalised to have mass one at each point. We denote the $L^p$ space of
functions on $G$ with respect to Haar measure by $L^p(G)$. As mentioned in
Example \reff{Banach Spaces}{L^p example}, we will not usually distinguish
between functions defined on $G$ and their $L^p$ equivalence classes.

Using Haar measure, we may turn $L^1(G)$ into a Banach algebra.

\begin{theorem}\label{convolution}
Let $G$ be an LCA group with Haar measure $dy$
and suppose $f,g\in L^1(G)$. Then for almost all $x\in G$, the function
$y\mapsto f(x-y)g(y)$ for $y\in G$ is integrable on $G$.
If we write
\[h(x)=\int_G f(x-y)g(y)dy,\]
then $h\in L^1(G)$ and $\norm{h}_1\leq\norm{f}_1\norm{g}_1$.
\end{theorem}

\begin{definition} Let $G,f,g$ be as in Theorem \reff{lca}{convolution}. Then
the {\em convolution} of $f$ and $g$, denoted $f*g$, is given by
\[(f*g)(x)=\int_G f(x-y)g(y)dy\]
for almost all $x\in G$
\end{definition}

\begin{corollary} Let $G$ be an LCA group. Then $L^1(G)$ is a
Banach Algebra under convolution and pointwise addition.
\end{corollary}

Our next aim is to define the Fourier transform of a function in $L^1(G)$.
We start by introducing the characters of $G$.

\begin{definition} A {\em character} on an LCA group $G$ is a continuous mapping
$\xi$ from $G$ into $\C$ such that $|\xi(x)|=1$
and $\xi(x+y)=\xi(x)\xi(y)$ for all $x,y\in G$. The set of all characters of
$G$ is denoted $\hatt{G}$.
\end{definition}

Thus a character is a continuous homomorphism of $G$ into $\T$. It can be shown
that the set of characters on $\T$ is the set $\{\varphi_n\}_{n\in\Z}$, where
$\varphi_n(t)=e^{int}$ for all $n\in\Z$ and all $t\in[0,2\pi]$. Every character
of
$\Z$ has the form $n\mapsto e^{int}$ for some $t\in[0,2\pi]$. The characters of
$\R$ are all of the form $x\mapsto e^{ixy}$ for some $y\in\R$.

For any LCA group $G$, the set of characters $\hatt{G}$ can be made
an Abelian group. We shall denote its group operation by $+$ (in what follows,
this should cause no confusion with the group operation $+$ for $G$) and
define it by $(\xi_1+\xi_2)(x)=\xi_1(x)\xi_2(x)$ for all $x\in G$.
We write $\ip<x,\xi>$ for $\xi(x)$.

A topology is defined on $\hatt{G}$ by specifying that
$\{\xi_n\}_{n=1}^{\infty}\subset\hatt{G}$ converges to $\xi\in\hatt{G}$ if
$\{\xi_n\}_{n=1}^{\infty}\subset\hatt{G}$ converges uniformly to $\xi$ on all
compact subsets of $G$. It can be shown that this topology turns $\hatt{G}$ into
a locally compact Abelian group (see \cite[Chapter VII, \S 3]{Katznelson}).
We say that $\hatt{G}$ is the {\em dual group}
of $G$.

The {\em Pontryagin Duality Theorem} (see \cite[p.189]{Katznelson}) asserts
that, for $x\in G$
and $\xi\in\hatt{G}$, the mapping $\xi\mapsto \ip<x,\xi>$ is a character on
$\hatt{G}$, and every character on $\hatt{G}$ is of this form. Moreover, the
topology induced by uniform convergence on compact subsets of $\hatt{G}$
coincides with the original topology on $G$. In otherwords, if $\hatt{G}$ is the
dual group of $G$, then $G$ is the dual of $\hatt{G}$.

We illustrate the Pontryagin Duality Theorem for the LCA group $\T$.
We saw that there exists a
bijection between $\hatt{\T}$ and $\Z$, since given any $\xi\in\hatt{\T}$, there
is a unique $n\in\Z$ such that $\ip<t,\xi>=e^{int}$ for all $t\in\T$.
Moreover, the topology $\hatt{T}$ is the
discrete topology. Thus $\hatt{\T}\simeq\Z$. One may similarly show that
$\hatt{\Z}\simeq\T$. This duality may be described (by abuse of notation) as
$\ip<n,t>=e^{int}=\ip<t,n>$ for
all $t\in\T$ and all $n\in\Z$. (On
the left hand side, we regard $n$ as an element of $\Z$ while on the right
hand side we regard it as a function on $\T$.)

We now have all the tools needed to define the Fourier transform of an
integrable
function on an LCA group.

\begin{definition}\label{Fourier transf}
Let $G$ be an LCA group. Then the {\em
Fourier transform} of a function $f\in L^1(G)$ is defined by
\[\hatt{f}(\xi) = \int_G \overline{\ip<y,\xi>}f(y)dy\]
for all $\xi\in\hatt{G}$, where $dy$ is Haar measure on $G$.
\end{definition}

Thus the Fourier transform of a function $f\in L^1(\Z)$ is
\[\hatt{f}(t)=\sum_{n=-\infty}^{\infty}e^{-int}f(n)\]
for all $t\in\T$, since Haar measure on $\Z$ is unit point-mass measure.
Similarly, the Fourier transform of a function $f\in L^1(\T)$ is
\[\hatt{f}(n) = \frac{1}{2\pi}\int_0^{2\pi}e^{-int}f(t)dt\]
for all $n\in\Z$, where $dt$ is Lebesgue measure. The {\em Fourier series} of
$f$ is the corresponding formal expression
\[\sum_{n=-\infty}^{\infty}\hatt{f}(n)\varphi_n\]
where $\varphi_n(t)=e^{int}$ for all $t\in\T$ and all $n\in\Z$.


We aim to use the Fourier transform to construct a large class of operators
which act on the $L^p$ space of a given LCA group. We may do this via {\em
Plancherel's Theorem}.

\begin{theorem}\label{isometry}\cite[Chapter VII, \S 4]{Katznelson}
{\em Plancherel's Theorem.} Let $G$ be a locally compact Abelian group. Then the
Fourier transform on $L^1(G)$ is an isometry of $L^1\cap L^2(G)$ onto a dense
subspace of $L^2(\hatt{G})$. Hence it can be extended to an isometry of $L^2(G)$
onto $L^2(\hatt{G})$.
\end{theorem}

For the circle group $\T$, Plancherel's theorem takes the following form.

\begin{theorem}\label{Plancherel for T}\cite[Theorem I.5.5]{Katznelson}

(i) Let $f\in L^2(\T)$. Then
\[\sum_{n=\infty}^{\infty}|\hatt{f}(n)|^2=
\frac{1}{2\pi}\int_0^{2\pi}|f(t)|^2\,dt,\]
that is, $\ssnorm{\hatt{f}}_{L^2(\Z)}=\norm{f}_{L^2(\T)}$.

(ii) For each $f\in L^2(\T)$,
\[f=\lim_{N\rightarrow\infty}\sum_{n=-N}^N\hatt{f}_n\varphi_n\]
in $L^2(\T)$ norm, where $\varphi_n:t\mapsto e^{int}$.

(iii) Given any sequence $\{a_n\}_{n=-\infty}^{\infty}$ of complex numbers in
$L^2(\Z)$, then there exists a unique $f\in L^2(\T)$ such that
$a_n=\hatt{f}(n)$.

Thus the correspondence $f\leftrightarrow\{\hatt{f}(n)\}_{n=-\infty}^{\infty}$
is an isometry between $L^2(\T)$ and $L^2(\Z)$.
\end{theorem}


A by-product of the above theorem is that the set $\{\varphi_n\}_{n\in\Z}$ of
functions in the Hilbert space $L^2(\T)$ forms a complete orthonormal system
(and
hence an unconditional basis) with respect to the inner product
\[\ip<f,g>=\frac{1}{2\pi}\int_0^{2\pi}f(t)\overline{g(t)}\,dt.\]

It is harder to establish whether or not $\{\varphi_n\}_{n\in\Z}$ is a basis for
$L^p(\T)$ when $1\leq p<\infty$ and $p\neq2$. Standard results about Fej\'{e}r's
kernel (see Section \ref{lca} and \cite[I.2.6]{Katznelson}) give the following
facts. First recall that a {\em trigonometric polynomial} is a function on
$\T$ of
the form $\sum_{n=-N}^Na_n\varphi_n$ where $N\in\N$ and $a_n\in\C$.

\begin{theorem}\label{Fejer facts}
Fix $1\leq p<\infty$. Then

(i) the trigonometric polynomials are dense in $L^p(\T)$;

(ii) if $f,g\in L^p(\T)$ and $\hatt{f}(n)=\hatt{g}(n)$ for all $n\in\Z$,
then $f=g$; and


(iii) if the Fourier series of a function $f\in L^p(\T)$ does converge in
$L^p$-norm, then it converges to $f$.
\end{theorem}

So if the Fourier series does converge for each function in $L^p(\T)$,
$\{\varphi_n\}_{n\in\Z}$ is a basis for $L^p(\T)$. Otherwise,
one would like to know whether or not $\{\varphi_n\}_{n\in\Z}$ is a basic
sequence in $L^p(\T)$ for $p\neq2$, and whether this sequence is unconditional.
One way to study such problems is through multiplier transforms.

\begin{definition}\label{multipliers}
Let $G$ be an LCA group and let
$\phi:\hatt{G}\rightarrow\C$ be bounded and measurable on $\hatt{G}$. By
Plancherel's Theorem, define $S_{\phi}$ to be the continuous linear mapping of
$L^2(G)$ into itself for which
$({S_{\phi}f})\,\hatt{\,} = \phi \hatt{f}$ for all $f\in L^2(G)$.
Then for $1\leq p\leq\infty$, $\phi$ is said to be an
{\em $L^p(G)$ multiplier} if
and only if there is a constant $C>0$ such that
\begin{equation}\label{eq-multiplier norm}
\norm{S_{\phi}f}_p\leq C_p\norm{f}_p
\end{equation}
for all $f\in L^2\cap L^p(\T)$.
In this case $S_{\phi}$ is called the {\em multiplier transform corresponding
to $\phi$ on $L^p(G)$} and $\phi$ may also be referred to as a
{\em Fourier multiplier} or a {\em multiplier function} for $L^p(G)$.
\end{definition}


The space of Fourier multipliers for $L^p(G)$ will be denoted by
$M_p(\hatt{G})$. We turn $M_p(\hatt{G})$ into a Banach space by equipping it
with
the following norm: for $\phi\in M_p(\hatt{G})$ define
$\norm{\phi}_{M_p(\hatt{G})}$ to be the usual operator norm on $L^p(G)$ for the
operator $S_{\phi}$. Thus $\norm{\phi}_{M_p(\hatt{G})}$ is the smallest
possible $C\geq 0$ for which (\ref{eq-multiplier norm}) holds. It is
trivial to check from the definitions that if $\phi_1,\phi_2\in M_p(\hatt{G})$,
then the product $\phi_1\phi_2$, defined by pointwise operations, is in
$M_p(\hatt{G})$ and
\[\norm{\phi_1\phi_2}_{M_p(\hatt{G})}\leq\norm{\phi_1}_{M_p(\hatt{G})}
\norm{\phi_2}_{M_p(\hatt{G})}.\]
In fact, we have the following result.

\begin{proposition}\label{multiplier algebra}\cite[\S3]{BG Spectral}
Let $G$ be an LCA group. Then for $1\leq p<\infty$,
$M_p(\hatt{G})$ is a Banach algebra under pointwise operations, and the mapping
$\phi\mapsto S_{\phi}$ is an identity-preserving algebra homomorphism of
$M_p(\hatt{G})$ into $\B(L^p(G))$.
\end{proposition}

The next example illustrates some of the concepts that
have been introduced so far.

\begin{example}\label{lca example}
Let $G=\T$ and let $\varphi_n:t\mapsto e^{int}$ for all $t\in\T$ and $n\in\Z$.
Fix an $m\in\Z$
and consider the characteristic function $\chi_{\{m\}}$ on $\Z$.
Then for any $f\in L^2(\T)$ we have, by Definition \reff{lca}{multipliers},
\[\bigl(S_{\chi_{\{m\}}}f\bigr)\,\hatt{\,}\,(n)=\chi_{\{m\}}(n)\hatt{f}(n)
=\chi_{\{m\}}(n)\hatt{f}(m)=\hatt{\varphi}_m(n)\hatt{f}=
\bigl(\hatt{f}(m)\varphi_m\bigr)\,\hatt{\,}\,(n)\]
for all $n\in\Z$. Thus $S_{\chi_{\{m\}}}f=\hatt{f}(m)\varphi_m$, and
$S_{\chi_{\{m\}}}$ projects each $f$ onto the $m$th term of its Fourier
series.

Now we consider the convolution product of $f$ with $\varphi_m$.
For almost all $t\in\T$ we have
\begin{eqnarray*}
(\varphi_m*f)(t) &=& \frac{1}{2\pi}\int_0^{2\pi}\varphi_m(t-s)f(s)ds \\
& = & e^{imt}\,\frac{1}{2\pi}\int_0^{2\pi}e^{-ims}f(s)ds \\
& = & e^{imt}\hatt{f}(m).
\end{eqnarray*}
Thus $S_{\chi_{\{m\}}}$ and the operator $f\mapsto\varphi_m*f$ coincide on
$L^2(\T)$. It is easy to see that $\ssnorm{S_{\chi_{\{m\}}}}=1$ and hence
$\ssnorm{\chi_{\{m\}}}_{M_p(\hatt{G})}=1$.
\end{example}

We now give one reason why multiplier transforms are of interest to us.
Let $\varphi_n(t)=e^{int}$ for each $n\in\Z$ and $t\in\T$. One might hope that
$\{\varphi_n\}_{n\in\Z}$ is a basis for $L^p(\T)$. Assume for the moment that
this is the case. The expansion of a
function $f\in L^p(\T)$ with respect to this basis is then given by
$\sum_{n\in\Z}\hatt{f}(n)\varphi_n$.
If one wanted to prove that the basis was unconditional, it would suffice to
find (by Proposition \reff{bases}{unconditional conv}) a constant $C_p$
depending only on $p$ such that
\[\snorm{\sum_{n\in\Z}\varepsilon_n\hatt{f}(n)\varphi_n}_p\leq C_p\norm{f}_p\]
for all $f\in L^p(\T)$ and all choices
$\varepsilon_n=\pm 1$. This would be
equivalent to finding a constant $C_p$ such that
\[\norm{S_{\varepsilon}f}_p\leq C_p\norm{f}_p\]
for all $f\in L^2\cap L^p(\T)$ and all functions $\varepsilon:\Z\rightarrow\C$
taking values in $\{-1,1\}$. Motivated by this, we develop the general theory of
multipliers a little furt
