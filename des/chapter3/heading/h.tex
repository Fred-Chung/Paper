%%%%%%%%%%%%%%%%%%%%%%%%%%%%%%%%%%%%%%%%%%%%%%%%%%%%%%%%%%%%%%%%%%%%
\chapter{Classical Harmonic Analysis}\label{cha}

%%%%%%%%%%%%%%%%%%%%%%%%%%%%%%%%%%%%%%%%%%%%%%%%%%%%%%%%%%%%%%%%%%%%
We now
leave the broad discussion about bases and conditionality in Banach spaces and
instead focus our attention to some particular classes of the $L^p$ Banach
spaces. The theory of classical harmonic analysis provides a good starting point
from which to tackle problems relating to bases and conditionality in this
setting.

In this chapter we give an overview of some basic results from classical
harmonic analysis and Fourier theory.
We begin by introducing the fundamental concepts of harmonic analysis on locally
compact Abelian groups. The Fourier transform and Fourier multipliers will be
defined. Connections between multiplier transforms and convolution operators
will
be drawn in Section \ref{lca}, as well as important examples given. In
Section \ref{lca} we discuss some of the classical results of M. Riesz,
Littlewood--Paley and Marcinkiewicz.

The general theory given in this chapter can be used to consider the
the set of functions $\{\varphi_n\}_{n\in\Z}\subset L^p(\T)$, where
$\varphi_n:t\mapsto e^{int}$. Is this a basis for the Banach space $L^p(\T)$?
Is it an unconditional basic sequence? If not, how close is it
to being an unconditional sequence? The results stated in Section
\ref{lca}
answer such questions.
